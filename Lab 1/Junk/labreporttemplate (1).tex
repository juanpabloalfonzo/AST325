\documentclass{article}
\usepackage[margin=1.0in]{geometry}
\usepackage{graphicx}

\author{Name 1 \\ johnny@appleseed.com}
\title{Hello Lab Report!}
\date{October 1, 2016}

\begin{document}

\maketitle

\begin{abstract}
This is my abstract!
\end{abstract}

\section{Introduction}

I can introduce the lab and its purpose. I will talk about my group members here. 

\section{Equipment}

This is the equipment I've used. I can put this in a list: 

\begin{enumerate}
\item A Doodad
\item A Whatsit
\item A Whosit
\item A length of string
\end{enumerate}

\section{Data Acquisition}

Here's how I'll talk about my experiment \& how I went about it. Maybe
I'll put a figure in (like figure \ref{fig1}). 

\begin{figure}
\begin{center}
\includegraphics[scale=0.2]{telescope.png}
\caption{This is a telescope! \label{fig1}}
\end{center}
\end{figure}

\section{Observations}

I can explain my measurements and observations here and put a table in here!

\begin{table}

\begin{tabular}{l|lcr}
Day & Mubdi's breakfast & Mubdi's Lunch & Mubdi's Dinner \\
\hline
Monday & cow & dog & cat \\
Tuesday & kangaroo & koala & dingo \\
Wednesday & polar bear & ostrich & kangaroo \\
Thursday & hamster & rat & black bear \\
\hline
\end{tabular}

\caption{This is my simple data table.}
\end{table}

\section{Data Reduction}

And maybe some equations? How about an inline equation with a couple
of greek characters $\alpha_{\lambda} \approx \cdots $. Or maybe a
numbered equation that I can refer to:

\begin{equation}
\frac{3}{4} = \infty
\end{equation}

Or maybe I want to prove something with an equation array (or eqnarray):

\begin{eqnarray}
\frac{3}{1000} & = & \frac{\lambda}{q} \\
0.003 & = & \frac{\lambda}{q}
\end{eqnarray}

\section{Discussion}

I can discuss my results here. 

\section{Conclusions}

And come up with my conclusions.

\end{document}
